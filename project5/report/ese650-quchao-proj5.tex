%----------------------------------------------------------------------------------------
%	PACKAGES AND OTHER DOCUMENT CONFIGURATIONS
%----------------------------------------------------------------------------------------

\documentclass[twoside]{article}

\usepackage{lipsum} % Package to generate dummy text throughout this template

\usepackage[sc]{mathpazo} % Use the Palatino font
\usepackage[T1]{fontenc} % Use 8-bit encoding that has 256 glyphs
\linespread{1.05} % Line spacing - Palatino needs more space between lines
\usepackage{microtype} % Slightly tweak font spacing for aesthetics

\usepackage[hmarginratio=1:1,top=32mm,columnsep=20pt]{geometry} % Document margins
\usepackage{multicol} % Used for the two-column layout of the document
\usepackage[hang, small,labelfont=bf,up,textfont=it,up]{caption} % Custom captions under/above floats in tables or figures
\usepackage{booktabs} % Horizontal rules in tables
\usepackage{float} % Required for tables and figures in the multi-column environment - they need to be placed in specific locations with the [H] (e.g. \begin{table}[H])
\usepackage{hyperref} % For hyperlinks in the PDF

\usepackage{lettrine} % The lettrine is the first enlarged letter at the beginning of the text
\usepackage{paralist} % Used for the compactitem environment which makes bullet points with less space between them

\usepackage{abstract} % Allows abstract customization
\renewcommand{\abstractnamefont}{\normalfont\bfseries} % Set the "Abstract" text to bold
\renewcommand{\abstracttextfont}{\normalfont\small\itshape} % Set the abstract itself to small italic text

\usepackage{titlesec} % Allows customization of titles
\renewcommand\thesection{\Roman{section}} % Roman numerals for the sections
\renewcommand\thesubsection{\Roman{subsection}} % Roman numerals for subsections
\titleformat{\section}[block]{\large\scshape\centering}{\thesection.}{1em}{} % Change the look of the section titles
\titleformat{\subsection}[block]{\large}{\thesubsection.}{1em}{} % Change the look of the section titles

\usepackage{fancyhdr} % Headers and footers
\pagestyle{fancy} % All pages have headers and footers
\fancyhead{} % Blank out the default header
\fancyfoot{} % Blank out the default footer
\fancyhead[C]{ESE650 Learning in Robotics $\bullet$ April 2014 $\bullet$ Project 5} % Custom header text
\fancyfoot[RO,LE]{\thepage} % Custom footer text

\usepackage[pdftex]{graphicx}
\usepackage{epstopdf}
\usepackage{subfigure}
\usepackage{amsmath,amssymb,amsopn,amstext,amsfonts}
\usepackage{url}
\usepackage[usenames,dvipsnames]{color}
\usepackage{siunitx}
\usepackage{amsmath}
\usepackage{amsfonts}
\usepackage{amssymb}

\graphicspath{{fig/}}
\newcommand{\W}{\mathcal{W}}
\newcommand{\X}{\mathcal{X}}
\newcommand{\Y}{\mathcal{Y}}
\newcommand{\Z}{\mathcal{Z}}
\newcommand{\red}[1]{\textcolor{red}{#1}}
\newcommand{\brown}[1]{\textcolor{brown}{#1}}
%----------------------------------------------------------------------------------------
%	TITLE SECTION
%----------------------------------------------------------------------------------------

\title{\vspace{-15mm}\fontsize{24pt}{10pt}\selectfont\textbf{Cost Learning and Path Planning}} % Article title

\author{
\large
\textsc{Chao Qu}\thanks{A thank you or further information}\\[2mm] % Your name
\normalsize University of Pennsylvania \\ % Your institution
\normalsize \href{mailto:quchao@seas.upenn.edu}{quchao@seas.upenn.edu} % Your email address
\vspace{-5mm}
}
\date{}

%----------------------------------------------------------------------------------------

\usepackage{graphicx}
\begin{document}


\maketitle % Insert title

\thispagestyle{fancy} % All pages have headers and footers

%----------------------------------------------------------------------------------------
%	ABSTRACT
%----------------------------------------------------------------------------------------

%\begin{abstract}
%
%\noindent Hey, I'm just an abstract. % Dummy abstract text
%
%\end{abstract}

%----------------------------------------------------------------------------------------
%	ARTICLE CONTENTS
%----------------------------------------------------------------------------------------

%----------------------------------------------------------------------------------------
%	INTRODUCTION
%----------------------------------------------------------------------------------------

\begin{multicols}{2} % Two-column layout throughout the main article text

\section{Introduction}
\lettrine[nindent=0em,lines=2]{G}esture recognition based on inertial sensor is the main topic in this project. Out task is to classify gestures based on imu data collected from a mobile phone.

%----------------------------------------------------------------------------------------
%	PRE-PROCESSING
%----------------------------------------------------------------------------------------

\section{Pre-processing}
%----------------------------------------------------------------------------------------
%	METHODS
%----------------------------------------------------------------------------------------

\section{Methods}

%----------------------------------------------------------------------------------------
%	RESULTS
%----------------------------------------------------------------------------------------

\section{Results}
%----------------------------------------------------------------------------------------
%	DISCUSSION
%----------------------------------------------------------------------------------------

\section{Discussion}

%----------------------------------------------------------------------------------------
%	REFERENCE LIST
%----------------------------------------------------------------------------------------

\begin{thebibliography}{2} % Bibliography - this is intentionally simple in this template

%\bibitem{Vesa00}
%Vesa-Matti Mantyla, \emph{Hand gesture recognition of a mobile device user}. Multimedia and Expo, 2000. ICME 2000.
%
%\bibitem{Elmez07}
%Elmezain M. and Al-Hamadi A., \emph{A Hidden Markov Model-based continuous gesture recognition system for hand motion trajectory}. 19th International Conference on Pattern Recognition, ICPR 2008.
%
%\bibitem{Rabiner89}
%Lawrence R. Rabiner, \emph{A Tutorial on Hidden Markov Models and Selected Applications in Speech Recognition}. Proceedings of the IEEE, 1989.

\end{thebibliography}

%----------------------------------------------------------------------------------------

\end{multicols}
\begin{table}[H]
\centering
\begin{tabular}{|c|c|c|c|c|c|c|c|}\hline
Data & circle & figure8 & fish & hammer & pend & wave & Result \\ \hline\hline
circle 1 & -11.18 & -72.04 & -29.73 & -37.82 & -19.55 & NaN    & circle \\
circle 2 & -0.53  & -57.85 & -21.51 & -30.59 & -11.87 & -19.13 & circle \\
circle 3 & -0.53  & -58.13 & -21.61 & -30.74 & -11.93 & -19.22 & circle \\
circle 4 & -20.21 & -83.98 & -40.41 & -43.26 & -26.17 & NaN    & circle \\
circle 5 & -0.58  & -62.16 & -23.11 & -32.87 & -12.75 & -20.55 & circle \\ \hline \hline
figure8 1 & NaN & -215.96 & NaN & NaN & NaN & NaN & figure8 \\
figure8 2 & NaN & -492.84 & NaN & NaN & NaN & NaN & figure8 \\
figure8 3 & NaN & -223.82 & NaN & NaN & NaN & NaN & figure8 \\
figure8 4 & NaN & -235.13 & NaN & NaN & NaN & NaN & figure8 \\
figure8 5 & NaN & -282.01 & NaN & NaN & NaN & NaN & figure8 \\ \hline\hline
fish 1 & NaN & -402.64 &  -71.99 & -661.56 & NaN & -300.02 & fish \\
fish 2 & NaN & -355.60 & -100.07 & -555.12 & NaN & -279.19 & fish \\
fish 3 & NaN & -294.87 & -160.96 &     NaN & NaN &     NaN & fish \\
fish 4 & NaN & -413.77 &  -65.99 & -695.41 & NaN & -327.80 & fish \\
fish 5 & NaN & -363.73 &  -40.79 & -612.45 & NaN & -274.36 & fish \\ \hline\hline
hammer 1 & NaN & -247.01 & NaN & -170.56 & NaN & NaN & hammer\\
hammer 2 & NaN & -251.22 & NaN & -193.95 & NaN & NaN & hammer\\
hammer 3 & NaN & -238.77 & NaN & -160.98 & NaN & NaN & hammer\\
hammer 4 & NaN & -237.53 & NaN & -168.17 & NaN & NaN & hammer\\
hammer 5 & NaN & -267.45 & NaN & -174.82 & NaN & NaN & hammer\\ \hline\hline
pend 1 & -1086.39 & -1028.01 & -1135.85 & -452.05 & -3.16 & NaN & pend\\
pend 2 &  -950.61 &  -899.42 &  -993.27 & -395.63 & -3.16 & NaN & pend\\
pend 3 &  -730.17 &  -712.95 &  -772.32 & -315.53 & -7.11 & NaN & pend\\
pend 4 &  -846.52 &  -800.83 &  -883.97 & -352.37 & -3.16 & NaN & pend\\
pend 5 & -1222.16 & -1156.61 & -1278.42 & -508.47 & -3.16 & NaN & pend\\\hline\hline
wave 1 & NaN & -245.71 & -386.53 & NaN & NaN & -175.85 & wave\\
wave 2 & NaN & -296.00 & -536.27 & NaN & NaN & -236.02 & wave\\
wave 3 & NaN & -179.59 & -280.66 & NaN & NaN & -122.30 & wave\\
wave 4 & NaN & -222.42 & -463.23 & NaN & NaN & -171.95 & wave\\
wave 5 & NaN & -310.66 & -572.36 & NaN & NaN & -246.30 & wave\\\hline
\end{tabular}
\caption{Training and validation results}
\label{table:result}
\end{table}

\end{document}
