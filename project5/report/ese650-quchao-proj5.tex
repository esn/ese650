%----------------------------------------------------------------------------------------
%	PACKAGES AND OTHER DOCUMENT CONFIGURATIONS
%----------------------------------------------------------------------------------------

\documentclass[twoside]{article}

\usepackage{lipsum} % Package to generate dummy text throughout this template

\usepackage[sc]{mathpazo} % Use the Palatino font
\usepackage[T1]{fontenc} % Use 8-bit encoding that has 256 glyphs
\linespread{1.05} % Line spacing - Palatino needs more space between lines
\usepackage{microtype} % Slightly tweak font spacing for aesthetics

\usepackage[hmarginratio=1:1,top=32mm,columnsep=20pt]{geometry} % Document margins
\usepackage{multicol} % Used for the two-column layout of the document
% Custom captions under/above floats in tables or figures
\usepackage[hang, small,labelfont=bf,up,textfont=it,up]{caption}
\usepackage{booktabs} % Horizontal rules in tables
\usepackage{float}
\usepackage{hyperref} % For hyperlinks in the PDF

\usepackage{lettrine}
\usepackage{paralist}

\usepackage{abstract} % Allows abstract customization
\renewcommand{\abstractnamefont}{\normalfont\bfseries} % Set the "Abstract" text to bold
\renewcommand{\abstracttextfont}{\normalfont\small\itshape} % Set the abstract itself to small italic text

\usepackage{titlesec} % Allows customization of titles
\renewcommand\thesection{\Roman{section}} % Roman numerals for the sections
\renewcommand\thesubsection{\Roman{subsection}} % Roman numerals for subsections
\titleformat{\section}[block]{\large\scshape\centering}{\thesection.}{1em}{}
\titleformat{\subsection}[block]{\large}{\thesubsection.}{1em}{}

\usepackage{fancyhdr} % Headers and footers
\pagestyle{fancy} % All pages have headers and footers
\fancyhead{} % Blank out the default header
\fancyfoot{} % Blank out the default footer
\fancyhead[C]{ESE650 Learning in Robotics $\bullet$ April 2014 $\bullet$ Project 5}
\fancyfoot[RO,LE]{\thepage} % Custom footer text

\usepackage[pdftex]{graphicx}
\usepackage{epstopdf}
\usepackage{subfigure}
\usepackage{amsmath,amssymb,amsopn,amstext,amsfonts}
\usepackage{url}
\usepackage[usenames,dvipsnames]{color}
\usepackage{siunitx}
\usepackage{amsmath}
\usepackage{amsfonts}
\usepackage{amssymb}

\graphicspath{{fig/}}
\newcommand{\W}{\mathcal{W}}
\newcommand{\X}{\mathcal{X}}
\newcommand{\Y}{\mathcal{Y}}
\newcommand{\Z}{\mathcal{Z}}
\newcommand{\red}[1]{\textcolor{red}{#1}}
\newcommand{\brown}[1]{\textcolor{brown}{#1}}

%-------------------------------------------------------------------------------
%	TITLE SECTION
%-------------------------------------------------------------------------------

\title{\vspace{-15mm}\fontsize{24pt}{10pt}\selectfont\textbf{Cost Learning and Path Planning}} % Article title

\author{
\large
\textsc{Chao Qu}\thanks{A thank you or further information}\\[2mm] % Your name
\normalsize University of Pennsylvania \\ % Your institution
\normalsize \href{mailto:quchao@seas.upenn.edu}{quchao@seas.upenn.edu} % Your email address
\vspace{-5mm}
}
\date{}

%-------------------------------------------------------------------------------

\usepackage{graphicx}
\begin{document}


\maketitle % Insert title

\thispagestyle{fancy} % All pages have headers and footers

%-------------------------------------------------------------------------------
%	ABSTRACT
%-------------------------------------------------------------------------------

%\begin{abstract}
%
%\noindent Hey, I'm just an abstract. % Dummy abstract text
%
%\end{abstract}

%-------------------------------------------------------------------------------
%	ARTICLE CONTENTS
%-------------------------------------------------------------------------------

%-------------------------------------------------------------------------------
%	INTRODUCTION
%-------------------------------------------------------------------------------

\begin{multicols}{2} % Two-column layout throughout the main article text

\section{Introduction}
\lettrine[nindent=0em,lines=2]{P}lanning.

%-------------------------------------------------------------------------------
%	PRE-PROCESSING
%-------------------------------------------------------------------------------

\section{Pre-processing}
%-------------------------------------------------------------------------------
%	METHODS
%-------------------------------------------------------------------------------

\section{Methods}

%-------------------------------------------------------------------------------
%	RESULTS
%-------------------------------------------------------------------------------

\section{Results}
%-------------------------------------------------------------------------------
%	DISCUSSION
%-------------------------------------------------------------------------------

\section{Discussion}

%-------------------------------------------------------------------------------
%	REFERENCE LIST
%-------------------------------------------------------------------------------

\begin{thebibliography}{2} % Bibliography - this is intentionally simple in this template

%\bibitem{Vesa00}
%Vesa-Matti Mantyla, \emph{Hand gesture recognition of a mobile device user}. Multimedia and Expo, 2000. ICME 2000.
%
%\bibitem{Elmez07}
%Elmezain M. and Al-Hamadi A., \emph{A Hidden Markov Model-based continuous gesture recognition system for hand motion trajectory}. 19th International Conference on Pattern Recognition, ICPR 2008.
%
%\bibitem{Rabiner89}
%Lawrence R. Rabiner, \emph{A Tutorial on Hidden Markov Models and Selected Applications in Speech Recognition}. Proceedings of the IEEE, 1989.

\end{thebibliography}

%-------------------------------------------------------------------------------

\end{multicols}

\end{document}
